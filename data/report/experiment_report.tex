\documentclass[sigconf]{acmart}

\AtBeginDocument{ \providecommand\BibTeX{ Bib\TeX } }
\setcopyright{acmlicensed}
\copyrightyear{2025}
\acmYear{2025}
\acmDOI{XXXXXXX.XXXXXXX}

\acmConference[BI 2025]{Business Intelligence}{-}{-}

\begin{document}

\title{BI2025 Experiment Report - Group 06}
%% ---Authors: Dynamically added ---

          \author{Nikola Lukic}
          \authornote{Student A, Matr.Nr.: 12127674}
          \affiliation{
            \institution{TU Wien}
            \country{Austria}
          }
          
          \author{Kerim Halilovic}
          \authornote{Student B, Matr.Nr.: 12434665}
          \affiliation{
            \institution{TU Wien}
            \country{Austria}
          }
          

\begin{abstract}
  This report documents the machine learning experiment for Group 06, following the CRISP-DM process model.
\end{abstract}

\ccsdesc[500]{Computing methodologies~Machine learning}
\keywords{CRISP-DM, Provenance, Knowledge Graph, Machine Learning}

\maketitle

%% --- 1. Business Understanding ---
\section{Business Understanding}

\subsection{Data Source and Scenario}
The dataset consists of two CSV files (`rome\_weekdays.csv` and `rome\_weekends.csv`) containing Airbnb listings in Rome. It includes attributes such as price (`realSum`), room type, capacity, cleanliness rating, guest satisfaction, and distance from the city center and metro.

Scenario: A property management firm in Rome wants to optimize their pricing strategy and investment portfolio. They need to understand which factors most significantly drive listing prices (e.g., is being closer to the metro more valuable than a high cleanliness rating?) to predict the optimal price for new properties entering the market.

\subsection{Business Objectives}
The primary business objective is to maximize revenue for property owners by setting optimal rental prices. This requires identifying the key drivers of listing value in the Rome market (e.g., location vs. amenities) and providing a tool to estimate fair market value for new or existing listings to avoid underpricing or overpricing.

%% --- 2. Data Understanding ---
\section{Data Understanding}
\textbf{Dataset Description:} 

The following features were identified in the dataset:

\begin{table}[h]
  \caption{Raw Data Features}
  \label{tab:features}
  \begin{tabular}{lp{0.2\linewidth}p{0.4\linewidth}}
    \toprule
    \textbf{Feature Name} & \textbf{Data Type} & \textbf{Description} \\
    \midrule
    
    \bottomrule
  \end{tabular}
\end{table}

%% --- 3. Data Preparation ---
\section{Data Preparation}
\subsection{Data Cleaning}
Describe your Data preparation steps here and include respective graph data.


\end{document}
