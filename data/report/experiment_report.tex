\documentclass[sigconf]{acmart}

\AtBeginDocument{ \providecommand\BibTeX{ Bib\TeX } }
\setcopyright{acmlicensed}
\copyrightyear{2025}
\acmYear{2025}
\acmDOI{XXXXXXX.XXXXXXX}

\acmConference[BI 2025]{Business Intelligence}{-}{-}

\begin{document}

\title{BI2025 Interim Experiment Report - Group 06}
%% ---Authors: Dynamically added ---

          \author{Nikola Lukic}
          \authornote{Student A, Matr.Nr.: 12127674}
          \affiliation{
            \institution{TU Wien}
            \country{Austria}
          }
          
          \author{Kerim Halilovic}
          \authornote{Student B, Matr.Nr.: 12434665}
          \affiliation{
            \institution{TU Wien}
            \country{Austria}
          }
          

\begin{abstract}
  This report documents the interim results (Phases 1-3) of the machine learning experiment for Group 06, following the CRISP-DM process model.
\end{abstract}

\ccsdesc[500]{Computing methodologies~Machine learning}
\keywords{CRISP-DM, Provenance, Knowledge Graph, Machine Learning}

\maketitle

%% --- 1. Business Understanding ---
\section{Business Understanding}

\subsection{Data Source and Scenario}
The dataset consists of two CSV files (`rome\_weekdays.csv` and `rome\_weekends.csv`) containing Airbnb listings in Rome. It includes attributes such as price (`realSum`), room type, capacity, cleanliness rating, guest satisfaction, and distance from the city center and metro.

Scenario: A property management firm in Rome wants to optimize their pricing strategy and investment portfolio. They need to understand which factors most significantly drive listing prices (e.g., is being closer to the metro more valuable than a high cleanliness rating?) to predict the optimal price for new properties entering the market.

\subsection{Business Objectives}
The primary business objective is to maximize revenue for property owners by setting optimal rental prices. This requires identifying the key drivers of listing value in the Rome market (e.g., location vs. amenities) and providing a tool to estimate fair market value for new or existing listings to avoid underpricing or overpricing.

%% --- 2. Data Understanding ---
\section{Data Understanding}
\textbf{Dataset Description:} Weekdays and Weekends Airbnb listings in Rome with prices and distances.

The following features were identified in the dataset:

\begin{table}[h]
  \caption{Raw Data Features}
  \label{tab:features}
  \begin{tabular}{lp{0.2\linewidth}p{0.4\linewidth}}
    \toprule
    \textbf{Feature Name} & \textbf{Data Type} & \textbf{Description} \\
    \midrule
    bedrooms & double> & The number of bedrooms in the listing. \\
    biz & boolean> & Whether the listing is for business purposes or not. \\
    cleanliness\_rating & double> & The cleanliness rating of the listing. \\
    dist & double> & The distance from the city centre. \\
    guest\_satisfaction\_overall & double> & The overall guest satisfaction rating of the listing. \\
    host\_is\_superhost & boolean> & Whether the host is a superhost or not. \\
    is\_weekend & boolean> & Boolean flag indicating if data is from weekend file \\
    lat & double> & The latitude of the listing. \\
    lng & double> & The longitude of the listing. \\
    metro\_dist & double> & The distance from the nearest metro station. \\
    multi & boolean> & Whether the listing is for multiple rooms or not. \\
    person\_capacity & double> & The maximum number of people that can stay in the room. \\
    realSum & double> & The total price of the Airbnb listing. \\
    room\_private & boolean> & Whether the room is private or not. \\
    room\_shared & boolean> & Whether the room is shared or not. \\
    room\_type & string> & The type of room being offered (e.g. private, shared, etc.). \\
    \bottomrule
  \end{tabular}
\end{table}

%% --- 3. Data Preparation ---
\section{Data Preparation}
\subsection{Data Cleaning and Transformation}
The following data preparation steps were performed:

\begin{itemize}
    \item \textbf{Transform Features} Applied data transformations:\newline 1. Log-Transformation: Applied log-plus-one transformation to realSum to address the high skewness identified in data understanding.\newline 2. Feature Selection: Excluded attr\_index and rest\_index variants due to lack of documentation and high correlation with distance.\newline 3. Encoding: Applied One-Hot Encoding to room\_type.\newline 4. Scaling: Applied Standard Scaler to numeric features to normalize magnitudes.
    \item \textbf{3.b Analyze Pre-processing Alternatives} 
    \item \textbf{3.c Analyze Derived Attributes} 
    \item \textbf{3.d Analyze External Data Sources} 
    \item \textbf{Transform Features} Applied data transformations:\newline 1. Log-Transformation: Applied log-plus-one transformation to realSum to address the high skewness identified in data understanding.\newline 2. Feature Selection: Excluded attr\_index and rest\_index variants due to lack of documentation and high correlation with distance.\newline 3. Encoding: Applied One-Hot Encoding to room\_type.\newline 4. Scaling: Applied Standard Scaler to numeric features to normalize magnitudes.
    \item \textbf{3.b Analyze Pre-processing Alternatives} 
    \item \textbf{3.c Analyze Derived Attributes} 
    \item \textbf{3.d Analyze External Data Sources} 
    \item \textbf{Transform Features} Applied data transformations:\newline 1. Log-Transformation: Applied log-plus-one transformation to realSum to address the high skewness identified in data understanding.\newline 2. Feature Selection: Excluded attr\_index and rest\_index variants due to lack of documentation and high correlation with distance.\newline 3. Encoding: Applied One-Hot Encoding to room\_type.\newline 4. Scaling: Applied Standard Scaler to numeric features to normalize magnitudes.
    \item \textbf{3.b Analyze Pre-processing Alternatives} 
    \item \textbf{3.c Analyze Derived Attributes} 
    \item \textbf{3.d Analyze External Data Sources} 
\end{itemize}

%% --- Future Work ---
\section{Future Work}
\textit{Modeling, Evaluation, and Deployment will be covered in the final submission.}

\end{document}
